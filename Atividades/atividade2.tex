\documentclass{codeclass}

\title{Atividade 2}
\author{\texorpdfstring{$\Jimeens\limits_{({\color{red!70!black}11917239})}$}{}}
\affiliation{Instituto de Física, \href{https://ror.org/036rp1748}{Universidade de São Paulo}, Rua do Matão 1371, 05508-090, Cidade Universitária, São Paulo, Brasil} % Affiliation
\dated{\today}

\abstract{Considere a base de dados contida no arquivo \verb|Stars.cvs|, disponível para download na página da disciplina no Moodle, sobre as características observadas de um conjunto de 240 estrelas.
Elabore um código em Python para o processamento desses dados de forma a responder as seguintes questões

\rule{\linewidth}{1pt}

Comentário: Criei um repositório no Github para salvar os dados e também salvei um arquivo nomeado \verb|Atividade1.ipynb| também no meu GitHub, contendo as mesmas informações e comentários presentes neste PDF, então para tornar a correção possivelmente mais simples, os arquivos estão contidos nos \textit{shields} abaixo, basta clicar em qual arquivo é mais simples de verificar

\begin{center}
    \drawBadge[ labelColor=black,
                color=red!70!black, 
                logo=\colab, link=https://colab.research.google.com/github/Jimeens/Mestrado/blob/main/Atividade1.ipynb]
                {}{Abrir No Colab}
    \drawBadge[ labelColor=black,
                color=red!70!black, 
                logo=\faGithub, link=https://github.com/Jimeens/Mestrado/blob/daf75c8e525f3bc504d9162be78b9395a4d4eacd/Aprendizado\%20de\%20M\%C3\%A1quina\%20e\%20Intelig\%C3\%AAncia\%20Artificial\%20em\%20F\%C3\%ADsica/Atividade1/Atividade1.ipynb]
                {}{Abrir no GitHub}
\end{center}}

\begin{document}

\maketitle

\section{Análise exploratória}

    Vamos primeiro importar os dados que iremos utilizar. Como o arquivo que contém os dados é relativamente grande (69.6 MB), ele foi adicionado no drive e importado no Google Colab usando a biblioteca \verb|gdown|. Os dados originais serão armazenados na variável \verb|dados|.
\begin{longlisting}
    \begin{minted}{python}
    import gdown
    import pandas as pd

    file = {
        'filename': 'HIGGS_100k.csv',
        "file_id" : "1mrzlmpPh4copC7ZULocAGhA1xAep88Sy"
    }

    url = f"https://drive.google.com/uc?id={file['file_id']}"
    gdown.download(url, file['filename'], quiet=True)
    print(f'{file['filename']} baixado com sucesso e armazenado em "dados".')

    dados = pd.read_csv('HIGGS_100k.csv')
    \end{minted}
\end{longlisting}
\begin{verbatim}
    HIGGS_100k.csv baixado com sucesso e armazenado em "dados".
\end{verbatim}

Antes de começar, vamos renomear os atributos apropriadamente, dado que eles estão originalmente nomeados como números, o que torna a análise mais complicada. Como informado na proposta da atividade, as colunas em ordem são: \verb|class label|, \verb|lepton pT|, \verb|lepton eta|, \verb|lepton phi|, \verb|missing energy magnitude|, \verb|missing energy phi|, \verb|jet 1 pt|, \verb|jet 1 eta|, \verb|jet 1 phi|, \verb|jet 1 b-tag|, \verb|jet 2 pt|, \verb|jet 2 eta|, \verb|jet 2 phi|, \verb|jet 2 b-tag|, \verb|jet 3 pt|, \verb|jet 3 eta|, \verb|jet 3 phi|, \verb|jet 3 b-tag|, \verb|jet 4 pt|, \verb|jet 4 eta|, \verb|jet 4 phi|, \verb|jet 4 b-tag|, \verb|m_jj|, \verb|m_jjj|, \verb|m_lv|, \verb|m_jlv|, \verb|m_bb|, \verb|m_wbb| e \verb|m_wwbb|, portanto serão estes os nomes atribuidos a cada variável.
\begin{longlisting}
    \begin{minted}{python}
    names = {
    '1.000000000000000000e+00': 'class label',
    '8.692932128906250000e-01': 'lepton pT',
    '-6.350818276405334473e-01': 'lepton eta',
    '2.256902605295181274e-01': 'lepton phi',
    '3.274700641632080078e-01': 'missing energy magnitude',
    '-6.899932026863098145e-01': 'missing energy phi',
    '7.542022466659545898e-01': 'jet 1 pt',
    '-2.485731393098831177e-01': 'jet 1 eta',
    '-1.092063903808593750e+00': 'jet 1 phi',
    '0.000000000000000000e+00': 'jet 1 b-tag',
    '1.374992132186889648e+00': 'jet 2 pt',
    '-6.536741852760314941e-01': 'jet 2 eta',
    '9.303491115570068359e-01': 'jet 2 phi',
    '1.107436060905456543e+00': 'jet 2 b-tag',
    '1.138904333114624023e+00': 'jet 3 pt',
    '-1.578198313713073730e+00': 'jet 3 eta',
    '-1.046985387802124023e+00': 'jet 3 phi',
    '0.000000000000000000e+00.1': 'jet 3 b-tag',
    '6.579295396804809570e-01': 'jet 4 pt',
    '-1.045456994324922562e-02': 'jet 4 eta',
    '-4.576716944575309753e-02': 'jet 4 phi',
    '3.101961374282836914e+00': 'jet 4 b-tag',
    '1.353760004043579102e+00': 'm_jj',
    '9.795631170272827148e-01': 'm_jjj',
    '9.780761599540710449e-01': 'm_lv',
    '9.200048446655273438e-01': 'm_jlv',
    '7.216574549674987793e-01': 'm_bb',
    '9.887509346008300781e-01': 'm_wbb',
    '8.766783475875854492e-01': 'm_wwbb',
    }

    dados.rename(columns=names, inplace=True)
    \end{minted}
\end{longlisting}

Como procedimento padrão, verifiquemos se existem dados faltantes e analisemos as características principais de cada um dos atributos.

\begin{longlisting}
    \begin{minted}{python}
    dados.info()
    \end{minted}
\end{longlisting}
\begin{verbatim}
    <class 'pandas.core.frame.DataFrame'>
    RangeIndex: 100000 entries, 0 to 99999
    Data columns (total 29 columns):
    #   Column                    Non-Null Count   Dtype  
    ---  ------                    --------------   -----  
    0   class label               100000 non-null  float64
    1   lepton pT                 100000 non-null  float64
    2   lepton eta                100000 non-null  float64
    3   lepton phi                100000 non-null  float64
    4   missing energy magnitude  100000 non-null  float64
    5   missing energy phi        100000 non-null  float64
    6   jet 1 pt                  100000 non-null  float64
    7   jet 1 eta                 100000 non-null  float64
    8   jet 1 phi                 100000 non-null  float64
    9   jet 1 b-tag               100000 non-null  float64
    10  jet 2 pt                  100000 non-null  float64
    11  jet 2 eta                 100000 non-null  float64
    12  jet 2 phi                 100000 non-null  float64
    13  jet 2 b-tag               100000 non-null  float64
    14  jet 3 pt                  100000 non-null  float64
    15  jet 3 eta                 100000 non-null  float64
    16  jet 3 phi                 100000 non-null  float64
    17  jet 3 b-tag               100000 non-null  float64
    18  jet 4 pt                  100000 non-null  float64
    19  jet 4 eta                 100000 non-null  float64
    20  jet 4 phi                 100000 non-null  float64
    21  jet 4 b-tag               100000 non-null  float64
    22  m_jj                      100000 non-null  float64
    23  m_jjj                     100000 non-null  float64
    24  m_lv                      100000 non-null  float64
    25  m_jlv                     100000 non-null  float64
    26  m_bb                      100000 non-null  float64
    27  m_wbb                     100000 non-null  float64
    28  m_wwbb                    100000 non-null  float64
    dtypes: float64(29)
    memory usage: 22.1 MB
\end{verbatim}

\begin{longlisting}
    \begin{minted}{python}
    dados.describe()
    \end{minted}
\end{longlisting}
\begin{table}[H]
    \centering
    \begin{tabular}{cccccccc}
        \toprule
         & \verb|class label| & \verb|lepton pT| & \verb|lepton eta| & $\cdots$ & \verb|m_bb| & \verb|m_wbb| & \verb|m_wwbb|  \\ 
        \midrule
        count & 100000.0000 & 100000.0000 & 100000.0000 & $\cdots$ & 100000.0000 & 100000.0000 & 100000.0000 \\
        mean & 0.528330 & 0.990366 & -0.003806 & $\cdots$ & 0.973076 & 1.031874 & 0.959203 \\
        std & 0.499199 & 0.561840 &	1.004840 & $\cdots$ & 0.523557 & 0.363395 & 0.313258 \\
        min & 0.000000 & 0.274697 & -2.434976 & $\cdots$ & 0.048125 & 0.303350 & 0.350939 \\
        25\% & 0.000000 & 0.590936 & -0.741244 & $\cdots$ & 0.673789 & 0.819170 & 0.769964 \\
        50\% & 1.000000 & 0.854835 & -0.002976 & $\cdots$ & 0.874004 & 0.947037 & 0.871038 \\
        75\% & 1.000000 & 1.236776 & 0.735292 & $\cdots$ & 1.139816 & 1.139032 & 1.057479 \\
        max & 1.000000 & 7.805887 & 2.433894 & $\cdots$ & 11.994177 & 7.318191 & 6.015647 \\
        \bottomrule
    \end{tabular}
\end{table}

É evidente que todos os dados estão em uma mesma escala de valores, então a priori não será necessário nenhum tipo de reescalonamento. O foco é utilizar apenas os atributos entitulados \textit{low-level}, então separaremos os 29 atributos em 3 conjuntos: o que contém apenas a variável alvo, que é a \verb|class label|, o que contém os 21 atributos \textit{low-level} e os 7 restantes que são os atributos \textit{high-level}.
\begin{longlisting}
    \begin{minted}{python}
    classLabel = dados.iloc[:, 0]
    lowLevel = dados.iloc[:, 1:22]
    highLevel = dados.iloc[:, 22:29]
    \end{minted}
\end{longlisting}

Como são muitos dados (100.000), é interessante verificar se eles estão balanceados, isto é, se a quantidade de dados com \verb|class label == 1| e semelhante à quantidade de dados com \verb|class label == 0|.
\begin{longlisting}
    \begin{minted}{python}
    print("Distribuição das classes:")
    print(classLabel.value_counts(normalize=True) * 100)  # Em percentual    
    \end{minted}
\end{longlisting}
\begin{verbatim}
    Distribuição das classes:
    class label
    1.0    52.833
    0.0    47.167
    Name: proportion, dtype: float64
\end{verbatim}

Temos então um conjunto de dados bem balanceado, como esperado, dado que estes foram utilizados no artigo original, com $\approx 53\%$ dos dados classificados como sinal  (\verb|class label == 1|) e $\approx 47\%$ classificados como fundo (\verb|class label == 0|). Podemos então separar os dados em conjuntos de treino, validação e teste para treinar a rede neural. Faremos uma separação de $60\%$ para treino, validação $15\%$ e $25\%$ para teste. E para reprodutibilidade, utilizaremos um \verb|random_state == 42|.
\begin{longlisting}
    \begin{minted}{python}
    from sklearn.model_selection import train_test_split
    import numpy as np

    trainRatio = 0.6
    validationRatio = 0.15
    testRatio = 0.25

    randomState = 42

    X = lowLevel
    y = classLabel

    XTrain, XTest, yTrain, yTest = train_test_split(X, y, test_size = 1 - trainRatio, stratify = y, random_state = randomState)
    XVal, XTest, yVal, yTest = train_test_split(XTest, yTest, test_size = testRatio/(testRatio + validationRatio), stratify = yTest, random_state = randomState)
    \end{minted}
\end{longlisting}

Uma verificação importante a se fazer com os dados é sobre o comportamento dos atributos em relação às ocorrências, isto é, a frequência de eventos. Para o conjunto de dados de treino que iremos utilizar (\verb|XTrain|), podemos fazer o plot dos 21 atributos \textit{low-level} com base na quantidade de dados separados em sinal e fundo, e verificar o comportamento destes em relação à frequência dos eventos.
\begin{longlisting}
    \begin{minted}{python}
    import matplotlib.pyplot as plt
    from matplotlib.lines import Line2D

    lowLevelWL = XTrain.copy()
    lowLevelWL['class label'] = classLabel

    fig, axs = plt.subplots(7, 3, figsize=(15, 35))
    axs = axs.flatten()

    for i, feature in enumerate(lowLevel.columns):
        data1 = lowLevelWL[lowLevelWL['class label'] == 1][feature]
        data0 = lowLevelWL[lowLevelWL['class label'] == 0][feature]

        min_all = min(data1.min(), data0.min())
        max_all = max(data1.max(), data0.max())

        bins = np.linspace(min_all, max_all, 51)

        axs[i].hist(data1, histtype='step', linewidth=2, color='black', label='Sinal (1)', bins=bins, density=False)
        axs[i].hist(data0, histtype='step', linewidth=2, color='red', linestyle='dotted', label='Fundo (0)', bins=bins, density=False)

        axs[i].set_title(f'Distribuição de {feature}')
        axs[i].set_xlabel(feature)
        axs[i].set_ylabel('Frequência de Eventos')
        axs[i].legend()

        legend_elements = [
            Line2D([0], [0], color='black', linewidth=2, label='Sinal (1)'),
            Line2D([0], [0], color='red', linewidth=2, linestyle='dotted', label='Fundo (0)')
        ]
        axs[i].legend(handles=legend_elements)

        axs[i].set_xlim(min_all, max_all)

    plt.tight_layout()
    plt.show()
    \end{minted}
\end{longlisting}
\begin{figure}[H]
    \centering
    \includegraphics[width = 0.9\linewidth]{figures/D1.png}
\end{figure}

\begin{figure}[H]
    \centering
    \includegraphics[width = 0.9\linewidth]{figures/D2.png}
    \includegraphics[width = 0.9\linewidth]{figures/D3.png}
    \includegraphics[width = 0.9\linewidth]{figures/D4.png}
    \includegraphics[width = 0.9\linewidth]{figures/D5.png}
\end{figure}

\begin{figure}[H]
    \centering
    \includegraphics[width = 0.9\linewidth]{figures/D6.png}
    \includegraphics[width = 0.9\linewidth]{figures/D7.png}
\end{figure}

Podemos ver então alguns comportamentos que se repetem em relação a alguns atributos (o que é de certa forma esperado, dado o nome de alguns deles). É possível então separar os atributos em 4 tipos de comportamento:
\begin{itemize}
    \item Comportamento similar ao de uma curva da forma $axe^{-x}$, com $a$ constante (não necessariamente essa função, mas muito similar):
    \begin{itemize}
        \item \verb|lepton pT|, \verb|missing energy magnitude|, \verb|jet 1 pt|, \verb|jet 2 pt|, \verb|jet 3 pt| e \verb|jet 4 pt|;
    \end{itemize}
    \item Comportamento mais uniforme uniforme, similar a uma curva $y=a$, com $a$ constante:
    \begin{itemize}
        \item \verb|lepton phi|, \verb|missing energy phi|, \verb|jet 1 phi|, \verb|jet 2 phi|, \verb|jet 3 phi| e \verb|jet 4 phi|;
    \end{itemize}
    \item Comportamento similar a uma gaussiana $ae^{-x^{2}}$, com $a$ constante:
    \begin{itemize}
        \item \verb|lepton eta|, \verb|jet 1 eta|, \verb|jet 2 eta|, \verb|jet 3 eta| e \verb|jet 4 eta|;
    \end{itemize}
    \item Comportamento de "picos":
    \begin{itemize}
        \item \verb|jet 1 b-tag|, \verb|jet 2 b-tag|, \verb|jet 3 b-tag| e \verb|jet 4 b-tag|.
    \end{itemize}
\end{itemize}

Estes comportamentos distintos implicam em cinemáticas distintas do conjunto de dados, permitindo mais informações físicas a serem interpretadas. É interessante agora verificar se os atributos em questão possuem alguma correlação forte, para caso exista, lidarmos com isto (mesmo que neste caso a correlação é inesperada, dados que os atributos são atributos físicos independentes).
\begin{longlisting}
    \begin{minted}{python}
    import seaborn as sns

    corr_matrix = XTrain.corr()

    sns.heatmap(corr_matrix, annot=False, cmap='inferno', vmin=-1, vmax=1)
    plt.title('Matriz de Correlação dos 21 atributos low-level')
    plt.show()
    \end{minted}
\end{longlisting}
\begin{figure}[H]
    \centering
    \includegraphics[width=0.8\linewidth]{figures/correlation21.png}
\end{figure}

Vemos então, conforme o esperado, que os atributos não possuem nenhuma correlação muito forte, o que faz com que não precisemos lidar com esse tipo de problema. Por fim, podemos verificar a presença de \textit{outliers} no conjunto de dados completo (se não houver no conjunto todo, logo não há nos conjuntos de treino, validação e teste). Consideraremos valores potencialmente descritos como \textit{outliers} aqueles cujo módulo da diferença entre o valor $x$ e a média $\mu$ forem maior do que 3 vezes o desvio padrão do atributo, isto é $|x − \mu| > 3\sigma$.
\begin{longlisting}
    \begin{minted}{python}
    lowDesc = lowLevel.describe()
    outliers = (lowLevel > lowDesc.loc['mean'] + 3 * lowDesc.loc['std']) | (lowLevel < lowDesc.loc['mean'] - 3 * lowDesc.loc['std'])
    print("Contagem de outliers potenciais por atributo:")
    print(outliers.sum())
    \end{minted}
\end{longlisting}
\begin{verbatim}
    Contagem de outliers potenciais por atributo:
    lepton pT                   1480
    lepton eta                     0
    lepton phi                     0
    missing energy magnitude    1279
    missing energy phi             0
    jet 1 pt                    1965
    jet 1 eta                      0
    jet 1 phi                      0
    jet 1 b-tag                    0
    jet 2 pt                    1615
    jet 2 eta                      0
    jet 2 phi                      0
    jet 2 b-tag                    0
    jet 3 pt                    1388
    jet 3 eta                      0
    jet 3 phi                      0
    jet 3 b-tag                    0
    jet 4 pt                    1481
    jet 4 eta                      0
    jet 4 phi                      0
    jet 4 b-tag                    0
    dtype: int64
\end{verbatim}

Levando em consideração que o conjunto de dados inteiro possui 100.000 dados, a quantidade de valores que podem possivelmente ser \textit{outliers} é muito menor, então podemos desconsiderar o tratamento de \textit{outliers} nos dados.

\section{Preparação dos dados: escalonamento, transformação, particionamento}

    Devido à alta diferença de escala nos dados da lista \verb|varNum|, é necessária a realização de um reescalonamento destas variáveis. Primeiro, podemos aplicar uma transformação logarítmica nas variáveis \verb|L| e \verb|R| devido à grande assimetria dos dados nelas contidos, onde somamos uma quantidade insignificante $1.0 \cdot 10^{-10}$ para evitar possíveis casos de $\log(0)$ (que não ocorrem aqui, mas é importante se considerar).
\begin{longlisting}
    \begin{minted}{py}
        dSNum = dS[varNum].copy()
        
        dSNum['L'] = np.log10(dSNum['L'] + 1e-10) 
        dSNum['R'] = np.log10(dSNum['R'] + 1e-10) 
    \end{minted}
\end{longlisting}

Em seguida, podemos de fato fazer o reescalonamento das variáveis, onde será usado o método \verb|StandardScaler()|, afim de padronizar a média em 0 e a variância em 1. A escolha deste método é devido à sua adequação para aplicação do PCA e da clusterização, lidando bem com distribuições não normais e possíveis \textit{outliers}.
\begin{longlisting}
    \begin{minted}{py}
        from sklearn.preprocessing import StandardScaler
        
        scaler = StandardScaler()
        scaledNum = scaler.fit_transform(dSNum)
        scaledNumdS = pd.DataFrame(scaledNum, columns = varNum)
        
        X = pd.concat([scaledNumdS, encodeddS], axis = 1)
    \end{minted}
\end{longlisting}
onde agora, \verb|X| é a tabela \verb|dS| com as variáveis numéricas reescalonadas e com as variáveis categóricas agora codificadas em valores binários, sendo então uma tabela com 240 linhas e 18 colunas
\begin{table}[H]
    \centering
    \begin{tabular}{ccccccc}
        \toprule
        \verb|index| & \verb|Temperature| & \verb|L| & \verb|R| & $\cdots$ & \verb|Spectral_Class_M| & \verb|Spectral_Class_O|  \\ 
        \midrule
        0 & -0.779382 & -0.888096 & -0.950995 & $\cdots$ & 1.0 & 0.0 \\
        1 & -0.782110 & -1.061480 & -0.974683 & $\cdots$ & 1.0 & 0.0 \\
        2 & -0.828477 & -1.117944 & 0.602446 & $\cdots$ & 1.0 & 0.0 \\
        3 & -0.807496 & -1.16276 & -0.965717 & $\cdots$ & 1.0 & 0.0 \\
        4 & -0.897819 & -1.203776 & 0.604815 & $\cdots$ & 1.0 & 0.0 \\
        $\cdots$ & $\cdots$ & $\cdots$ & $\cdots$ & $\cdots$ & $\cdots$ & $\cdots$ \\
        235 & 2.983743 & 1.197284 & 1.230755 & $\cdots$ & 0.0 & 1.0 \\
        236 & 2.133913 & 1.285690 & 1.199858 & $\cdots$ & 0.0 & 1.0 \\
        237 & -0.175029 & 1.237125 & 1.242467 & $\cdots$ & 0.0 & 0.0 \\
        238 & -0.132438 & 1.205825 & 1.182580 & $\cdots$ & 0.0 & 0.0 \\
        239 & 2.872754 & 1.170775 & 1.297235 & $\cdots$ & 0.0 & 1.0 \\
        \bottomrule
    \end{tabular}
\end{table}

% Afim de descobrir as possíveis correlações entre as variáveis, podemos utilizar a biblioteca \verb|seaborn| para construir um diagrama de correlação.
% \begin{longlisting}
%     \begin{minted}{py}
%         import seaborn as sns
        
%         matrizCorr = X.corr().round(2)
%         sns.heatmap(matrizCorr, cmap = 'RdBu', vmin = -1, vmax = 1);
%     \end{minted}
% \end{longlisting}
% \begin{figure}[H]
%     \centering
%     \includegraphics[width=.6\linewidth]{figures/correlation_heatmap.png}
% \end{figure}


\section{Avaliar a importância dos atributos com base nos coeficientes de um modelo de regressão linear múltipla (com ou sem regularização, à sua escolha)}

    Como discutido na parte 1, algumas variáveis possuem uma colinearidade considerável, o que sugere o uso de uma regressão linear com regularização L2 (Ridge), pois assim conseguimos evitar \textit{overfitting}. Para isso importamos a função \verb|Ridge| do \verb|sklearn|. Como o uso do Ridge exije a escolha de um parâmetro de regularização \verb|alpha|, podemos determinar a melhor escolha deste parâmetro utilizando o método de \textit{Grid Search} com validação crizada, usando o \verb|GridSearchCV| e usar como \verb|scoring| o atributo \verb|neg_mean_squared_error|, também do \verb|sklearn|, pois isso otimiza o erro quadrático médio na escala transformada \verb|yTEst|.
\begin{longlisting}
    \begin{minted}{python}
    from sklearn.linear_model import Ridge
    from sklearn.model_selection import GridSearchCV

    ridge = Ridge(random_state=42)
    paramGrid = {'alpha': [0.001, 0.01, 0.1, 1, 10, 100]}

    gridSearch = GridSearchCV(ridge, paramGrid, cv=5, scoring='neg_mean_squared_error', n_jobs=-1)
    gridSearch.fit(XTrain, yTrainLog)

    bestAlpha = gridSearch.best_params_['alpha']

    print("Melhor alpha:", bestAlpha)
    \end{minted}
\end{longlisting}
\begin{verbatim}
    Melhor alpha: 0.01
\end{verbatim}

Encontrado o melhor \verb|alpha|, basta aplicarmos o Ridge aos valores de treino \verb|XTrain| e \verb|yTrain|.
\begin{longlisting}
    \begin{minted}{python}
    ridge = Ridge(alpha=bestAlpha, random_state=42)
    ridge.fit(XTrain, yTrainLog)
    yPredRidge = ridge.predict(XTest)

    coeficientes = pd.Series(ridge.coef_, index = X.columns)
    print(coeficientes.sort_values(ascending = False).head(10))
    \end{minted}
\end{longlisting}
\begin{verbatim}
    wtd_mean_fie                 7.746846
    wtd_mean_atomic_radius       2.366502
    range_fie                    1.507192
    wtd_gmean_FusionHeat         1.365609
    mean_Valence                 1.005128
    wtd_gmean_atomic_mass        0.986664
    entropy_atomic_radius        0.865486
    wtd_mean_Density             0.822395
    gmean_fie                    0.821225
    wtd_entropy_atomic_radius    0.818379
    dtype: float64
\end{verbatim}

Para verificar o quão bom estão estes dados, podemos utilizar duas métricas: o MSE (\verb|mean_squared_error|) e o R² (\verb|r2_score|), ambos do sklearn.
\begin{longlisting}
    \begin{minted}{python}
    from sklearn.metrics import mean_squared_error, r2_score

    mseRidge = mean_squared_error(yTestLog, yPredRidge)
    r2Ridge = r2_score(yTestLog, yPredRidge)

    print("MSE:", mseRidge)
    print("R²:", r2Ridge)
    \end{minted}
\end{longlisting}
\begin{verbatim}
    MSE: 0.39034631146596926
    R²: 0.7663233037235521
\end{verbatim}

O valor de $R^{2} \approx 76.6\%$ indica que cerca de $76.6\%$ da variância da temperatura crítica (em escala logaritmica) pode ser explicada pelas variáveis do modelo. No caso do valor do erro médio quadrático, mesmo que seja um valor considerável, se analisarmos a raiz deste valor (RMSE), e exponenciarmos, teremos $\exp(\sqrt{\text{MSE}}) \approx 1.86$, indicando um erro multiplicativo médio inferior a $2$ na escala original, indicando que o modelo é capaz de capturar a maior parte das tendências que determinam a temperatura crítica. Fazemos isso, pois o valor obtido de MSE está na escala logaritmica, dado que transformamos a variável \verb|y|.

\section{Avaliar a importância dos atributos com base em um regressor Random Forest ou Gradient Boosting}

    Escolhendo um regressor Random Forest pra avaliar a importância dos atributos, importamos a função \verb|RandomForestRegressor| do \verb|sklearn|. Da mesma forma que fizemos para determinar o melhor \verb|alpha| na função Ridge, podemos determinar o melhor \verb|n_estimators| da função \verb|RandomForestRegressor| com o \verb|GridSearchCV|. Como o custo computacional pra encontrar esse valor otimizado é muito alto, é preferível definir uma profundidade máxima do Random Forest (que escolho como sendo 10).
\begin{longlisting}
    \begin{minted}{python}
    %%time
    from sklearn.ensemble import RandomForestRegressor

    RFR = RandomForestRegressor(random_state=42)
    paramGrid = { 'n_estimators': [150, 250, 350],
                'max_depth': [None, 20, 40],
                'min_samples_leaf': [1, 2, 4]}

    gridSearch = GridSearchCV(RFR, paramGrid, cv=3, scoring='neg_mean_squared_error', n_jobs=-1, verbose=1)
    gridSearch.fit(XTrain, yTrain)

    bestEstimators = gridSearch.best_params_['n_estimators']
    bestDepth = gridSearch.best_params_['max_depth']
    bestLeaf = gridSearch.best_params_['min_samples_leaf']

    print("Melhor n_estimators:", bestEstimators)
    print("Melhor max_depth:", bestDepth)
    print("Melhor min_samples_leaf:", bestLeaf)
    \end{minted}
\end{longlisting}
\begin{verbatim}
    Fitting 3 folds for each of 27 candidates, totalling 81 fits
    Melhor n_estimators: 250
    Melhor max_depth: 20
    Melhor min_samples_leaf: 1
    CPU times: user 6min 30s, sys: 13.8 s, total: 6min 44s
    Wall time: 3h 49min 2s
\end{verbatim}

Ao verificar os melhores hiperparâmetros selecionados para o Radom Forest, foi determinado após (3h 49min 2s) que o melhor valor de \verb|n_estimators = 350|, o melhor de \verb|max_depth = 20| e o melhor de \verb|min_samples_leaf = 1|. Como o código acima demora muito tempo pra ser executado, após confirmar os valores, torno essa parte do programa um comentário e salvo os melhores valores separadamente.
\begin{longlisting}
    \begin{minted}{python}
    bestEstimators = 350
    bestDepth = 20
    bestLeaf = 1
    \end{minted}
\end{longlisting}
\begin{longlisting}
    \begin{minted}{python}
    RFR = RandomForestRegressor(max_depth=bestDepth, min_samples_leaf=bestLeaf, n_estimators=bestEstimators, random_state=42)
    RFR.fit(XTrain, yTrainLog)
    yPredRFR = RFR.predict(XTest)

    importances = pd.Series(RFR.feature_importances_, index=X.columns)
    print(importances.sort_values(ascending=False).head(10))
    \end{minted}
\end{longlisting}
\begin{verbatim}
    range_ThermalConductivity        0.604150
    mean_Density                     0.052367
    wtd_gmean_ThermalConductivity    0.038381
    wtd_mean_atomic_mass             0.015584
    wtd_std_ElectronAffinity         0.015372
    wtd_gmean_ElectronAffinity       0.013949
    wtd_mean_ThermalConductivity     0.010111
    range_atomic_radius              0.009012
    std_ThermalConductivity          0.008445
    wtd_gmean_Valence                0.008275
    dtype: float64
\end{verbatim}

Aqui as variáveis obtidas pelo Random Forest vão ser aquelas que reduzem a variância na predição.
\begin{longlisting}
    \begin{minted}{python}
    mseRFR = mean_squared_error(yTestLog, yPredRFR)
    r2RFR = r2_score(yTestLog, yPredRFR)

    print("MSE:", mseRFR)
    print("R²:", r2RFR)
    \end{minted}
\end{longlisting}
\begin{verbatim}
    MSE: 0.11751767353522158
    R²: 0.9296492860335409
\end{verbatim}

Vemos então que $R^{2} \approx 93.0\%$ a variância da temperatura crícita (em escala logaritmica) pode ser explicada pelas variáveis selecionadas pelo modelo. No caso do erro médio quadrático, se determinarmos o RMSE e exponenciarmos, vamos obter algo como $\exp(\sqrt{\text{MSE}}) \approx 1.41$, indicando um erro multiplicativo ainda menor do que o obtido na regressão linear.

\section{Com base nos resultados dos itens 3 e 4, selecionar os atributos mais importantes. O número de atributos fica à sua escolha. Justificar sua escolha. Discutir brevemente se os atributos possuem significado físico, ou seja, se de fato pode existir uma relação com a variável alvo (temperatura crítica)}

    Com base nos itens 3 e 4, podemos selecionar os atributos mais importantes combinando os resultados e excluindo variáveis que possuem correlações muito altas entre si, isto é, se dentro dos atributos mais importantes temos atributos altamente correlacionados, apenas o atributo com maior valor de importância (normalizada) vai permanecer na análise. Isto faz com que possamos abranger um pouco mais da física do problema.

Como descrito no item 2, utilizamos um *threshold* de correlação de 0.6, então mantemos esse valor como base.
\begin{longlisting}
    \begin{minted}{python}
    corrThreshold = 0.6
    \end{minted}
\end{longlisting}

Podemos combinar os coeficientes do Ridge e as importâncias do Random Forest e normalizá-las.
\begin{longlisting}
    \begin{minted}{python}
    combinedImportances = pd.DataFrame({
        'RFRImportance': importances / importances.max(),
        'ridgeCoeficients': np.abs(coeficientes) / np.abs(coeficientes).max()
    }).fillna(0)
    \end{minted}
\end{longlisting}

Combinamos as métricas, utilizando uma média ponderada baseada no R² dos métodos.
\begin{longlisting}
    \begin{minted}{python}
    RFRWeight = r2RFR / (r2RFR + r2Ridge)
    ridgeWeight = r2Ridge / (r2RFR + r2Ridge)
    combinedImportances['combinedMetrics'] = RFRWeight * combinedImportances['RFRImportance'] + ridgeWeight * combinedImportances['ridgeCoeficients']
    \end{minted}
\end{longlisting}

Organizamos os atributos em ordem de importância e construimos uma matriz de correlação com o conjunto de treino.
\begin{longlisting}
    \begin{minted}{python}
    combinedImportancesSorted = combinedImportances.sort_values('combinedMetrics', ascending=False)
    corrMatrix = XTrain.corr().abs()
    \end{minted}
\end{longlisting}

Definimos uma função que remove variáveis altamente correlacionadas entre si levando em conta uma matriz de correlação, o valor das importâncias (sejam elas combinadas ou não) e o *threshold* pré estabelecido.
\begin{longlisting}
    \begin{minted}{python}
    def remove_correlated_features(corrMatrix, importances, threshold):
        selected = []
        excluded = set()

        for feature in importances.index:
            if feature in excluded:
                continue
            selected.append(feature)
            correlated = corrMatrix.index[corrMatrix[feature] > threshold].tolist()
            for c in correlated:
                if c != feature:
                    excluded.add(c)
        return selected
    \end{minted}
\end{longlisting}

Por fim, aplicamos esta função nas variáveis acima definidas e obtemos os atributos de interesse.
\begin{longlisting}
    \begin{minted}{python}
    selectedFeatures = remove_correlated_features(corrMatrix, combinedImportancesSorted, corrThreshold)

    print("Quantidade de atributos selecionados:", len(selectedFeatures))
    for f in selectedFeatures:
        print(f)
    \end{minted}
\end{longlisting}
\begin{verbatim}
    Quantidade de atributos selecionados: 14
    range_ThermalConductivity
    wtd_mean_fie
    wtd_mean_atomic_mass
    wtd_gmean_FusionHeat
    mean_fie
    std_atomic_mass
    range_Valence
    wtd_entropy_fie
    wtd_mean_ThermalConductivity
    std_ElectronAffinity
    wtd_gmean_ElectronAffinity
    wtd_std_FusionHeat
    entropy_ThermalConductivity
    wtd_range_Density
\end{verbatim}

Os atributos mais importantes identificados pelo modelo estão relacionados principalmente às propriedades térmicas, eletrônicas e estruturais dos materiais. Propriedades como condutividade térmica, afinidade e energia de ionização dos elétrons, massa atômica e valência influenciam diretamente a forma como os elétrons interagem com a rede cristalina. Essas interações determinam a energia necessária para a transição ao estado supercondutor, e portanto, contribuem com a temperatura crítica.

\section{Aplicar uma técnica de redução de dimensionalidade, como PCA (análise de componentes principais), criando novos atributos a partir de uma combinação dos atributos originais. O número de componentes principais a serem utilizadas fica à sua escolha (justificar escolha)}

    

Em relação ao método KMeans para agrupar as estrelas de acordo com suas características, é necessário escolher um estado alatório (\verb|random_state = |) para aplicação do KMeans, pois caso contrário, a análise não torna-se reprodutível, dado que para cada estado aleatório escolhido pelo próprio KMeans, a quantidade ótima de \textit{clusters} vai sempre mudar, prejudicando a análise. Portanto, podemos escolher um \verb|random_state = 2|\footnote{A escolha do \texttt{random\_state = 2} é feita com base na quantidade de tipos existentes na tabela \texttt{Categorias.csv}, pois a análise, \textit{a posteriori}, indica que o número ideal de \textit{clusters} é igual a 6. Foi estudada a possibilidade de diminuir a tolerância do KMeans (que por padrão é \texttt{0.0001}) para evitar que os gráficos se alterem a cada compilação, porém mesmo diminuindo até $10^{-20}$, sempre ocorria uma mudança no fator de silhueta, influenciando diretamente na quantidade de \textit{clusters} e portanto na análise.} na implementação do KMeans. Construímos então o fato de silhueta importando o \verb|KMeans| da biblioteca \verb|sklearn.cluster|.
\begin{longlisting}
    \begin{minted}{py}
        from sklearn.cluster import KMeans
        
        silKMeans = []
        for n in range(2, 11):
            kmeans = KMeans(n_clusters=n, random_state=2).fit(scaledNumdS)
            labels = kmeans.fit_predict(scaledNumdS)
            silKMeans.append(silhouette_score(scaledNumdS, labels))
        
        plt.plot(range(2,11), silKMeans, marker='o')
        plt.xlabel('Número de Clusters')
        plt.ylabel('Fator de Silhueta')
        plt.show()
    \end{minted}
\end{longlisting}
\begin{figure}[H]
    \centering
    \includegraphics[width=0.5\linewidth]{figures/KMeansSilhouette.png}
\end{figure}

Com isso, somos capazes de formar os \textit{clusters} definindo uma variável que armazena o valor máximo do fator de silhueta e aplicando \verb|n_clusters| igual a este valor.
\begin{longlisting}
    \begin{minted}{py}
        bestNumberClustersKMeans = list(range(2,11))[np.argmax(silKMeans)]

        kmeans = KMeans(n_clusters=bestNumberClustersKMeans).fit(scaledNumdS)
        categoriasKMeans = kmeans.labels_
        
        plt.figure(figsize=(15,10))
        for i, (xVar, yVar) in enumerate([('Temperature', 'L'), ('R','A_M'), ('Temperature','R'), ('L','A_M'), ('Temperature', 'A_M'), ('R','L')], 1):
            plt.subplot(2,3,i)
            plt.scatter(scaledNumdS[xVar], scaledNumdS[yVar], c=categoriasKMeans, cmap='inferno')
            plt.xlabel(xVar)
            plt.ylabel(yVar)
    \end{minted}
\end{longlisting}
\begin{figure}[H]
    \centering
    \includegraphics[width=1\linewidth]{figures/KMeans.png}
\end{figure}


\section{Construir um modelo de regressão linear múltipla com: a) os atribuutos mais importantes, escolhidos no item 5; b) usando as componentes principais como atributos (item 6). Comparar o desempenho desses modelos na predição da temperatura crítica e a sua inteerpretabilidade (isto é, se é fácil ou não interpretar o significado físico dos coeficientes ajustados). Avalie se o modelo é capaz de predizer diferentes faixas de valores de temperatura crítica}

    Com os atributos selecionados no item 5 (\verb|selectedFeatures|), podemos separar o conjunto de treino (\verb|XTrain|) e conjunto de teste (\verb|XTest|) selecionando essas variáveis.
\begin{longlisting}
    \begin{minted}{python}
    XTrainSelected = XTrain[selectedFeatures]
    XTestSelected = XTest[selectedFeatures]
    \end{minted}
\end{longlisting}

Fazendo então o modelo de regressão linear com regularização L2 (Ridge) com esses atributos.
\begin{longlisting}
    \begin{minted}{python}
    ridgeSelected = Ridge(alpha=bestAlpha, random_state=42)
    ridgeSelected.fit(XTrainSelected, yTrainLog)
    yPredRidgeSelected = ridgeSelected.predict(XTestSelected)

    mseRidgeSelected = mean_squared_error(yTestLog, yPredRidgeSelected)
    r2RidgeSelected = r2_score(yTestLog, yPredRidgeSelected)

    print("MSE:", mseRidgeSelected)
    print("R²:", r2RidgeSelected)
    \end{minted}
\end{longlisting}
\begin{verbatim}
    MSE: 0.579789840505915
    R²: 0.6529149360852038
\end{verbatim}

No caso das componentes do PCA, para evitar vazamento de dados, refazemos o PCA com 11 componentes no conjunto de treino (\verb|XTrain|) e no de teste (\verb|XTest|).
\begin{longlisting}
    \begin{minted}{python}
    XTrainPCA = pca.fit_transform(XTrain)
    XTestPCA = pca.transform(XTest)

    ridgePCA = Ridge(alpha=bestAlpha, random_state=42)
    ridgePCA.fit(XTrainPCA, yTrainLog)
    yPredRidgePCA = ridgePCA.predict(XTestPCA)

    mseRidgePCA = mean_squared_error(yTestLog, yPredRidgePCA)
    r2RidgePCA = r2_score(yTestLog, yPredRidgePCA)

    print("MSE:", mseRidgePCA)
    print("R²:", r2RidgePCA)
    \end{minted}
\end{longlisting}
\begin{verbatim}
    MSE: 0.5906987562390906
    R²: 0.6463844289083507
\end{verbatim}

Quando utilizamos todos os atributos, o $R^{2} \approx 76.6\%$, e nestes dois ultimos casos obtivemos $R^{2}_{\text{slctd}} \approx 65.3\%$ e $R^{2}_{\text{PCA}} \approx 64.6\%$, portanto, mesmo reduzindo bastante a dimensionalidade, os valores ainda são relativamente próximos, com uma diferença da ordem de $\sim 10\%$, o que é aceitável, tendo em mente que inicialmente tinhamos 81 atributos e reduzimos para 14 deles no primeiro método e 11 componentes principais no segundo.

No caso do resultado para os atributos selecionados ($R^{2}_{\text{slctd}}$), o significado físico é interpretável, pois estamos utilizando os atributos originais do problema, selecionando apenas os mais influentes. No caso das componentes principais do PCA ($R^{2}_{\text{PCA}}$), nós temos uma combinações lineares dos atributos originais, então interpretar fisicamente os coeficientes acaba sendo mais difícil.

Para verificar o quão bem os modelos predizem a temperatura crítica em diferentes faixas, podemos criar uma função que verifica o MSE em 3 diferentes regiões de temperatura, levando em conta que a temperatura crítica varia de $0.000210$ a $185~\text{K}$: abaixo de $60~\text{K}$, num intervalo entre $60$ e $120~\text{K}$ e acima de $120~\text{K}$.
\begin{longlisting}
    \begin{minted}{python}
    def mseValues(yTrue, yPred, yOriginal):
    temp = [0, 60, 120, np.inf]
    tempLabels = ['(<60K)', '(60-120K)', '(>120K)']
    groups = pd.cut(yOriginal, bins=temp, labels=tempLabels)

    for temps in tempLabels:
        mask = groups == temps
        if mask.sum() > 0:
        groupMSE = mean_squared_error(yTrue[mask], yPred[mask])
        print(f"MSE {temps}: {groupMSE}")
    \end{minted}
\end{longlisting}

Com essa função podemos avaliar o MSE para cada faixa de temperatura de cada um dos modelos usados.
\begin{longlisting}
    \begin{minted}{python}
    print("Ridge dos atributos selecionados:")
    mseValues(yTestLog, yPredRidgeSelected, yTest)

    print("\nRidge do PCA:")
    mseValues(yTestLog, yPredRidgePCA, yTest)
    \end{minted}
\end{longlisting}
\begin{verbatim}
    Ridge dos atributos selecionados:
    MSE (<60K): 0.6619997669538773
    MSE (60-120K): 0.33651629841528335
    MSE (>120K): 0.32663991238347967

    Ridge do PCA:
    MSE (<60K): 0.6732424357626846
    MSE (60-120K): 0.34564439110905343
    MSE (>120K): 0.35700521359657433
\end{verbatim}

É perceptível então que para faixas de temperatura mais altas, ambos os modelos possuem um erro multiplicativo mais baixo, porém para temperaturas menores o MSE é alto o suficiente para ser considerado problemático. Isso pode ser explicado possivelmente pela alta quantidade de materiais que possuem temperatura abaixo de $60~\text{K}$, conforme pode ser visto no histograma de temperatura crítica na parte 1, gerando uma alta variabilidade nos dados e talvez uma previsibilidade reduzida.

\section{Construir um modelo de regressão baseado em Random Forest ou Gradient Boosting com: a) apenas os atributos mais importantes, escolhidos no item 5; b) usando as componentes principais como atributos (item 6). Lembre-se de otimizar os hiperparâmetros. Comparar o desempenho desses modelos na predição da temperatura crítica e sua interpretabilidade. Avalie se o modelo é capaz de predizer diferentes faixas de valores de temperatura crítica}

    Fazendo o mesmo procedimento feito no item 7, porém utilizando Random Forest, precisamos otimizar alguns hiperparâmetros que serão utilizados, como foi feito no item 4, então usamos o GridSearchCV com os mesmos hiperparâmetros escolhidos no item 4, isto é
\begin{longlisting}
    \begin{minted}{python}
    paramGrid = { 'n_estimators': [100, 200, 350],
                'max_depth': [None, 20, 40],
                'min_samples_leaf': [1, 2, 4]}
    \end{minted}
\end{longlisting}

A partir disso, fazemos o o procedimento de encontrar esses hiperparâmetros. Como o código abaixo demorou 38min 58s, armazenamos os valores separadamente para execução mais rápida do que queremos de fato determinar, que será o $R^{2}_{\text{RFR}}$ e o MSE utilizando Random Forest.
\begin{longlisting}
    \begin{minted}{python}
    %%time
    RFRSelected = RandomForestRegressor(random_state=42)
    gridSearchSelected = GridSearchCV(RFRSelected, paramGrid, cv=3, scoring='neg_mean_squared_error', n_jobs=-1)
    gridSearchSelected.fit(XTrainSelected, yTrainLog)

    bestEstimatorsSelected = gridSearchSelected.best_params_['n_estimators']
    bestDepthSelected = gridSearchSelected.best_params_['max_depth']
    bestLeafSelected = gridSearchSelected.best_params_['min_samples_leaf']

    print("Melhor n_estimators:", bestEstimatorsSelected)
    print("Melhor max_depth:", bestDepthSelected)
    print("Melhor min_samples_leaf:", bestLeafSelected)
    \end{minted}
\end{longlisting}
\begin{verbatim}
    Melhor n_estimators: 350
    Melhor max_depth: 40
    Melhor min_samples_leaf: 1
    CPU times: user 1min 48s, sys: 0 ns, total: 1min 48s
    Wall time: 38min 58s
\end{verbatim}

\begin{longlisting}
    \begin{minted}{python}
    bestEstimatorsSelected = 350
    bestDepthSelected = 40
    bestLeafSelected = 1
    \end{minted}
\end{longlisting}

Com estes hiperparâmetros, podemos determinar os valores de MSE e $R^{2}_{\text{RFR}}$ com apenas os atributos selecionados no item 5.
\begin{longlisting}
    \begin{minted}{python}
    %%time
    RFRSelected = RandomForestRegressor(max_depth=bestDepthSelected, min_samples_leaf=bestLeafSelected, n_estimators=bestEstimatorsSelected, random_state=42)
    RFRSelected.fit(XTrainSelected, yTrainLog)
    yPredRFRSelected = RFRSelected.predict(XTestSelected)

    mseRFRSelected = mean_squared_error(yTestLog, yPredRFRSelected)
    r2RFRSelected = r2_score(yTestLog, yPredRFRSelected)

    print("MSE:", mseRFRSelected)
    print("R²:", r2RFRSelected)
    \end{minted}
\end{longlisting}
\begin{verbatim}
    MSE: 0.12665666417875918
    R²: 0.9241783258165404
    CPU times: user 1min 41s, sys: 0 ns, total: 1min 41s
    Wall time: 1min 41s
\end{verbatim}

Comparando os valores de $R^{2} \approx 93.0\%$ obtido com todos os atributos e o $R^{2}_{\text{RFR}} \approx 92.4\%$ obtido apenas com os atributos selecionados, temos uma diferença praticamente irrelevante neste valor, da ordem de $\sim 0.6\%$, portanto o Random Forest apenas com os atributos selecionados acaba sendo muito bom para prever a temperatura crítica.

Fazendo o mesmo procedimento para as componentes principais, temos primeiro otimizando os hiperparâmetros:
\begin{longlisting}
    \begin{minted}{python}
    %%time
    paramGrid = { 'n_estimators': [100, 200, 350],
                'max_depth': [None, 20, 40],
                'min_samples_leaf': [1, 2, 4]}

    RFRPCA = RandomForestRegressor(random_state=42)
    gridSearchPCA = GridSearchCV(RFRPCA, paramGrid, cv=3, scoring='neg_mean_squared_error', n_jobs=-1)
    gridSearchPCA.fit(XTrainPCA, yTrainLog)

    bestEstimatorsPCA = gridSearchPCA.best_params_['n_estimators']
    bestDepthPCA = gridSearchPCA.best_params_['max_depth']
    bestLeafPCA = gridSearchPCA.best_params_['min_samples_leaf']

    print("Melhor n_estimators:", bestEstimatorsPCA)
    print("Melhor max_depth:", bestDepthPCA)
    print("Melhor min_samples_leaf:", bestLeafPCA)
    \end{minted}
\end{longlisting}
\begin{verbatim}
    Melhor n_estimators: 350
    Melhor max_depth: 40
    Melhor min_samples_leaf: 1
    CPU times: user 1min 58s, sys: 68.9 ms, total: 1min 58s
    Wall time: 41min 27s
\end{verbatim}

Como o comando acima demora 41min 27s, os melhores parâmetros vão ser armazenados separadamente.
\begin{longlisting}
    \begin{minted}{python}
    bestEstimatorsPCA = 350
    bestDepthPCA = 40
    bestLeafPCA = 1
    \end{minted}
\end{longlisting}

E com isso o Random Forest com as componentes principais vai ficar:
\begin{longlisting}
    \begin{minted}{python}
    %%time
    RFRPCA = RandomForestRegressor(max_depth=bestDepthPCA, min_samples_leaf=bestLeafPCA, n_estimators=bestEstimatorsPCA, random_state=42)
    RFRPCA.fit(XTrainPCA, yTrainLog)
    yPredRFRPCA = RFRPCA.predict(XTestPCA)

    mseRFRPCA = mean_squared_error(yTestLog, yPredRFRPCA)
    r2RFRPCA = r2_score(yTestLog, yPredRFRPCA)

    print("MSE:", mseRFRPCA)
    print("R²:", r2RFRPCA)
    \end{minted}
\end{longlisting}
\begin{verbatim}
    MSE: 0.14267285887327238
    R²: 0.9145904000357674
    CPU times: user 1min 50s, sys: 275 ms, total: 1min 50s
    Wall time: 1min 51s
\end{verbatim}

Vemos então que em comparação, os valores de $R^{2} \approx 93.0\%$ e $R^{2}_{\text{PCA}} \approx 91.5\%$ são bem próximo, com uma diferença apenas da ordem de $\sim 1.5\%$, o que se mantém muito bom para previsibilidade dos valores de temperatura, porém interpretar fisicamente este caso acaba sendo inviável, por estarmos novamente tratando de uma combinação linear dos atributos mais relevantes.

Usando então a função \verb|mse_values| criada, podemos verificar a capacidade de predição dos 2 métodos em diferentes faixas de temperatura.
\begin{longlisting}
    \begin{minted}{python}
    print("Random Forest dos atributos selecionados:")
    mseValues(yTestLog, yPredRFRSelected, yTest)

    print("\nRandom Forest do PCA:")
    mseValues(yTestLog, yPredRFRPCA, yTest)
    \end{minted}
\end{longlisting}
\begin{verbatim}
    Random Forest dos atributos selecionados:
    MSE (<60K): 0.15579846994429303
    MSE (60-120K): 0.041285222868479916
    MSE (>120K): 0.014600577656779823

    Random Forest do PCA:
    MSE (<60K): 0.17247268155669102
    MSE (60-120K): 0.05577584675594284
    MSE (>120K): 0.01770229347152548
\end{verbatim}

Os valores de MSE para os dois métodos e para todas as faixas de temperatura consideradas foram bem baixas, o que é muito bom, pois o erro multiplicativo vai ser bem baixo para todas as faixas, sendo maior apenas na faixa onde tempo mais materiais que é para temperatura crítica $<60~\text{K}$.

\section{Fazer uma breve discussão crítica sobre o desempenho, a interpretabilidade e o custo computacional dos modelos lineares e dos modelos baseados em árvores de decisão}

    Ao olharmos para o fator desempenho dos modelos, temos que os modelos baseados em árvores de decisão, neste caso o Random Forest, são mais eficientes em relação à previsibilidade da temperatura crítica, porém o custo computacional é extremamente maior, mesmo considerando menos atributos. Quando otimizamos os hiperparâmetros do Random Forest utilizando o \verb|GridSerchCV| com todos os atributos, o custo computacional foi da ordem de $\sim 4$ horas, o que muitas vezes acaba sendo inviável, então reduzir a dimensionalidade dos atributos considerados em aproximadamente $4$ vezes reduziu esse custo quase em $4$ vezes também, mas ainda sendo muito mais custoso que um modelo linear de regressão.

Utilizando a redução de dimensionalidade das componentes principais, conseguimos predizer de forma consideravelmente boa os valores de temperatura crítica, porém acabamos perdendo a interpretabilidade física, pois cada componente principal vai utilizar de combinações lineares dos atributos originais diferentes, o que muitas vezes pode não ser muito bom, pois perdemos informações importantes, como o tipo de relação que um certo atributo terá com a temperatura crítica.

Podemos então concluir que os modelos de árvore de decisão são mais adequados para determinar a temperatura crítica dos materiais, principalmente considerando o caso onde consideramos apenas 14 dos 81 atributos originais, como feito no item 8.(a).

\end{document}