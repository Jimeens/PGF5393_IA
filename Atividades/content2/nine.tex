Ao olharmos para o fator desempenho dos modelos, temos que os modelos baseados em árvores de decisão, neste caso o Random Forest, são mais eficientes em relação à previsibilidade da temperatura crítica, porém o custo computacional é extremamente maior, mesmo considerando menos atributos. Quando otimizamos os hiperparâmetros do Random Forest utilizando o \verb|GridSerchCV| com todos os atributos, o custo computacional foi da ordem de $\sim 4$ horas, o que muitas vezes acaba sendo inviável, então reduzir a dimensionalidade dos atributos considerados em aproximadamente $4$ vezes reduziu esse custo quase em $4$ vezes também, mas ainda sendo muito mais custoso que um modelo linear de regressão.

Utilizando a redução de dimensionalidade das componentes principais, conseguimos predizer de forma consideravelmente boa os valores de temperatura crítica, porém acabamos perdendo a interpretabilidade física, pois cada componente principal vai utilizar de combinações lineares dos atributos originais diferentes, o que muitas vezes pode não ser muito bom, pois perdemos informações importantes, como o tipo de relação que um certo atributo terá com a temperatura crítica.

Podemos então concluir que os modelos de árvore de decisão são mais adequados para determinar a temperatura crítica dos materiais, principalmente considerando o caso onde consideramos apenas 14 dos 81 atributos originais, como feito no item 8.(a).