Fazer uma análise de regressão supervisionada para determinar a temperatura crítica de materiais. Utilizar diferentes modelos e subconjuntos de dados, comparando o desempenho e a interpretabilidade.

\rule{\linewidth}{1pt}

Comentário: Em um repositório no Github salvei um arquivo nomeado \verb|Atividade2.ipynb|, contendo as mesmas informações e comentários presentes neste PDF, então para tornar a correção possivelmente mais simples, os arquivos estão contidos nos \textit{shields} abaixo, basta clicar em qual arquivo é mais simples de verificar.

\begin{center}
    \drawBadge[ labelColor=black,
                color=red!70!black, 
                logo=\colab, link=https://colab.research.google.com/github/Jimeens/Mestrado/blob/main/Atividade2.ipynb]
                {}{Abrir No Colab}
    \drawBadge[ labelColor=black,
                color=red!70!black, 
                logo=\faGithub, link=https://github.com/Jimeens/Mestrado/blob/745d33d941c2c5366faf6c781c7e86a143d8083c/Aprendizado\%20de\%20M\%C3\%A1quina\%20e\%20Intelig\%C3\%AAncia\%20Artificial\%20em\%20F\%C3\%ADsica/Atividade2/Atividade2.ipynb]
                {}{Abrir no GitHub}
\end{center}